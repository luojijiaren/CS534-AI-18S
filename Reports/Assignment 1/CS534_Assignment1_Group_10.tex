\documentclass[11pt, a4paper]{article}
\usepackage{graphicx, fullpage, hyperref, listings}
\usepackage{appendix, pdfpages, color}
\usepackage{indentfirst} %段首空两格 棒
\usepackage{chngpage} 
\usepackage{tocloft}            % This squashes the Table of Contents a bit
\usepackage{pdfpages}
\usepackage{multirow}
\usepackage{amsmath}
\usepackage{framed}


\setlength\cftbeforesecskip{3pt}
\renewcommand{\contentsname}{\centerline{\textbf{Content}}}
\graphicspath{{images/}}

\usepackage{multicol}

\usepackage{graphicx}
\usepackage{epstopdf}
\hypersetup{CJKbookmarks,%
	bookmarksnumbered,%
	colorlinks,%
	linkcolor=black,%
	citecolor=black,%
	plainpages=false,%
	pdfstartview=FitH}

%%%%%%%代码语法高亮设置

\usepackage{color}

\definecolor{pblue}{rgb}{0.13,0.13,1}
\definecolor{pgreen}{rgb}{0,0.5,0}
\definecolor{pred}{rgb}{0.9,0,0}
\definecolor{pgrey}{rgb}{0.46,0.45,0.48}

\usepackage{listings}
\lstset{
	language=Java,
	showspaces=false,
	showtabs=false,
	%%%%%
	frame = single,
	stepnumber = 2,  
	numbersep = 4pt, 
	 numbers=left,
	%breakatwhitespace=false, 
	tabsize=2,  
	%%%%%
	breaklines=true,
	showstringspaces=false,
    breakatwhitespace=false, 
	commentstyle=\color{pgreen},
	keywordstyle=\color{pblue},
	stringstyle=\color{pred},
	basicstyle=\ttfamily,
	%moredelim=[il][\textcolor{pgrey}]{$$},
	%moredelim=[is][\textcolor{pgrey}]{\%\%}{\%\%},
}


%%%%%%%%代码语法高亮设置

\definecolor{MyLightYellow}{cmyk}{0,0.,0.2,0} 

\setlength{\parskip}{4pt}        % sets spacing between paragraphs
\interfootnotelinepenalty=500    % this prevents footnotes breaking across pages

\title{\includegraphics[width=0.45\textwidth]{wpi2}
        \\CS 534 Artificial Intelligence \\ Assignment 1 }          % <<<<<<<<< change the title as appropriate
\author{Group 10 }                    % <<<<<<<<< module code

\begin{document}
\begin{titlepage}
	
%\date{\today}
\maketitle
\addtocontents{toc}{\protect\thispagestyle{empty}} % because we don't want a page number on the title page
% Thanks to Huang Shanyue for suggesting this 

\begin{center}
Group Member
\end{center}

\begin{table}[htbp] 
\begin{center}
\begin{tabular}{l l l} 
	 
	 Yixuan & Jiao  &   yjiao@wpi.edu \\
     Yinkai & Ma  &   yma7@wpi.edu \\
     Jiaming & Nie  &  jnie@wpi.edu \\
     Pinyi & Xiao  &  pxiao@wpi.edu \\
\end{tabular}
\end{center}
\end{table}



%\date{\today}
\thispagestyle{empty}  %去除首页页码

\end{titlepage}

\tableofcontents
%\listoffigures

\section{N Queens Problem}

\subsection{Methodology}

\subsubsection{Hill Climbing Algorithm}



\subsubsection{A star Algorithm}


\subsection{Write Up Questions}

\subsubsection{The size of Puzzle}

\subsubsection{Effective Branching Factor}

\subsubsection{Cheaper Solutions}

\subsubsection{Solutions With Less Time Complexity}



\section{Urban Planning}

\subsection{Methodology}

\subsubsection{Hill Climbing}

\subsubsection{Genetic Algorithm}


\subsection{Write Up Question}

\subsubsection{Genetic Algorithm Mechanism}


\subparagraph{Selection}

The selection from a population is to select the individuals according to a probability sequence. The probability is determined by the score of each individual in a certain population. 

In a certain population, the higher score will lead to the higher probability to be fetched. 

\subparagraph{Crossover}

In this algorithm, crossover is using the the uniform crossover. For the gene segment on the chromosome, choose the segments randomly then perform the crossover.



\subparagraph{Elitism}

The elitism is performed before the culling and crossover operation. For the population in each iteration, remove the individual with the highest score 

\subparagraph{Culling}

\subparagraph{Mutation}

\subsubsection{Program Performance}


\subsubsection{Effects of Elitism and Culling}

\subparagraph{Effects of Elitism}

\subparagraph{Effects of Culling}

\subsubsection{Selection and Crossover}

\subparagraph{Perform of Selection}

\subparagraph{Perform of Crossover}

%\tableofcontents

%\listoffigures
%\listoftables
%\lstlistoflistings        


%\newpage




\bibliographystyle{IEEEtran}  
%\bibliography{MyRefs} 
%\addcontentsline{toc}{section}{References}





%-------------------------------------------------------------------------------------------------------





\end{document}
